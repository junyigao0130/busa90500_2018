\documentclass{article}
\usepackage{a4wide}
\usepackage{url}
\usepackage{hyperref}

\oddsidemargin -9mm
\evensidemargin -9mm
\textwidth 180mm
\textheight 240mm
\topmargin -20mm
\raggedright


\title{}
\date{}

\begin{document}

{\Large \bf
\begin{center}
MELBOURNE BUSINESS SCHOOL
(The University of Melbourne)

MBUSA90500 Programming, Module 2 2018
 
Assignment

Final submission before 9am 1 May 2018.

\end{center}
}
\section{Updates}

\begin{itemize}
    \item 18 April. The matrix begins with a row of zeros, so will have 21 rows at the end of the game.
    \item 18 April. Added {\tt "same\_col"} key to TEST dictionary
\end{itemize}
 
\section{Objectives}

This project is designed to give your syndicate experience in 
software engineering,
Python programming and get you thinking about data.
In particular, you should experience the following.
\begin{itemize}
    \item Python programming - every member of the syndicate must contribute code to the
          final solution.
    \item Software design - the process of breaking a task into modules and assigning 
          each module to a programmer in such a way that the pieces can be easily
          assembled.
    \item Software testing - each module should be tested prior to integration, and then
          the whole tested once integrated.
\end{itemize}



\section{Analytica Melburnia}

You are to write a virtual player for a two player game: Analytica Melburnia. 
You will submit your player to a tournament server and it will play you against the other syndicates.

The rules of the game are as follows.
\begin{enumerate}
    \item The game begins with a matrix of 5 columns and one row containing 5 zeros.
          Each column has a name that is randomly
          generated at the beginning of each game.
    \item Each game has 10 turns.
    \item At the beginning of each turn, both players chose 5 floating point numbers in the range $[-1023, 1023]$
          to add a row of the matrix.
    \item After the 10 turns, the matrix will have 20 rows.
    \item At the start of the game, each player is given a victory condition and a column name 
          that is only known to that player. See the list of victory conditions below.
    \item The winner of the game is the player that
has met their victory condition. If both or neither players have met their
condition, then the game is a draw.
\end{enumerate}

\subsection*{Victory Conditions}
\begin{description}
    \item[Max] The player's column must contain the unique maximum number in the matrix.
    \item[Min] The player's column must contain the unique minimum number in the matrix.
    \item[Linear] The player's column must have a statistically significant ($p < 0.05$) 
                  Pearson Correlation r of at least 0.9 when correlated with {\tt range(20)}.
    \item[Quadratic] The player's column squared must have a statistically significant ($p < 0.05$) 
                  Pearson Correlation r of at least 0.9 when correlated with {\tt range(20)}.
    \item[ZeroM] The player's column should have a mean whose absolute value is less than 0.000001.
    \item[SumNeg] The player's column should sum to less than zero.
    \item[SumPos] The player's column should sum to more than zero.
\end{description}



\section{Your player}

Your player must be Python3 code that defines a class called {\tt Player}
that contains a method called {\tt take\_turn}
which takes two parameters: the matrix so far in the form of a dictionary, and the victory condition as a tuple.

\begin{verbatim}
class Player:
    """Minimal player."""
    def take_turn(self, data, victory):
        """Must return a dictionary with the same keys as data, and with single float values 
           in the range [-1023, 1023].
           data is a dictionary with column names as keys and a list of floats 
                   as the column values of the game so far.
           victory the condition you should try and achieve and is a tuple (a,b) where 
                a is victory condition (string) 
                b is a column name (string - one of data.keys()).
        """
        return {k:1.0 for k in data}
\end{verbatim}

Your class can contain anything else you like, just as long as {\tt take\_turn} 
exists in the correct format.
You might like a {\tt \_\_repr\_\_} function.
There is no way to pass arguments to the constructor of your Player object in this tournament, so you 
should not define an {\tt \_\_init\_\_} function.


\section{Submitting to the tournament}

The tournament will run as a server expecting you to send commands
to add and delete players as JSON objects. You can ADD (and DEL) as many players as you
like to the tournament, so a syndicate could have multiple players in the tournament at
any one time.

The server is located at IP address 128.250.106.25 and port 5002.

You should use the following code (or similar) to send your JSON object to the server.

\begin{verbatim}
def send_to_server(js):
    """Open socket and send the json string js to server with EOM appended, and wait 
       for \n terminated reply.
       js - json object to send to server
    """
    clientsocket = socket.socket(socket.AF_INET, socket.SOCK_STREAM)
    clientsocket.connect(('128.250.106.25', 5002))

    clientsocket.send("""{}EOM""".format(js).encode('utf-8'))

    data = ''
    while data == '' or data[-1] != "\n":
        data += clientsocket.recv(1024).decode('utf-8')
    print(data)

    clientsocket.close()
\end{verbatim}

The JSON object you send should contain four key:value pairs as follows.
\begin{itemize}
\item Key {\tt "cmd"} which has the value of either {\tt "ADD"}, {\tt "DEL"} or {\tt "TEST"}.
\item Key {\tt "syn"} which has the value of your syndicate number.
\item Key {\tt "name"} which has the value of a team name for the leader board.
\item Key {\tt "data"} which has the value of string that is your python code.
\end{itemize}
If you use the {\tt "TEST"} command, you should also include the key:value pairs as follows.
\begin{itemize}
\item Key {\tt "data2"} which has the value of python code for the second player.
\item Key {\tt "vt1"} which has the value of the victory condition for player 1.
\item Key {\tt "vt2"} which has the value of the victory condition for player 2.
\item Key {\tt "same\_col"} which has the value of {\tt True} to force both players to use the same column. 
                            If {\tt False} will randomly choose columns.
\end{itemize}

Please delete your old players otherwise the system might grind to a halt.

You can view the leader board and get info about your player at 
\url{http://codemasters.eng.unimelb.edu.au/lb.html}

TEST is always between two players you provide.
An example TEST could be as follows.
\begin{verbatim}
p = "class Player:\n def take_turn(self, data, victory): return {k:1.1 for k in data}"

p2 = "class Player:\n def take_turn(self, data, victory): return {k:-1.1 for k in data}"

send_to_server(json.dumps({"cmd":"TEST", "syn":4, "name":"T4", "data":p, 
                           "data2":p2, "vt1":"SumPos", "vt2":"SumNeg", "same_col":True}))
\end{verbatim}

\section{Assessment}

\subsection{Within-syndicate responsibilities}

It is expected that you will structure the Python code so that it is
split into modules/functions, and different members of the syndicate
will write different parts of the code.
Naturally you can collaborate, but it is in the best interests of the weak
coders that they be given a coding task to complete.
The temptation for the strong coders to do the lot is high, but there
will be coding questions about this assignment on the exam that each
individual must answer.
I suggest the strong coders concentrate on the design, and splitting
the tasks up amongst the syndicate, aiming to allot the tasks
based on syndicate member's abilities.

\subsection{Submission}

The highest ranked player on the server from each syndicate as at
1 May 9am will be assessed.  I will get the code from the server,
and run it through a beautifier to format it and get a single file
of Python.  Please include comments, etc in your submission, and
feel free to include large comments on your approach to defeating
other players, and other strategic points, within the code.


\subsection{Marking Rubric for a Syndicate}

\renewcommand{\arraystretch}{2}

\begin{tabular}{|l|l|r|}
\hline
Component & Property & Mark \\
\hline
Submissions to server (socket + JSON)
            & At least one working player added &  /3 \\
(5 marks)   & Several players added and deleted &  /2 \\
\hline
Approach to trying to meet own victory conditions
            & Min, Max, SumPos, SumNeg, ZeroM &  /3\\
(6 marks)   & Linear, Quadratic               & /3\\
\hline
Approach to trying to defeat other's victory conditions
            & Min, Max              &  /1\\
(8 marks)   & SumPos, SumNeg, ZeroM &  /3\\
            & Linear, Quadratic     &  /4\\
\hline
Final position in tournament (3 marks) &  & /3\\
    $3 - \lfloor \mbox{\emph{position}}/4\rfloor$ & & \\
\hline
Code readability 
            & Variable names                       & /2\\
 (13 marks) & Major comments (eg functions)        & /5\\
            & Minor/in-line comments when required & /2\\
            & Modular design, no repeated code     & /4\\
\hline
 \multicolumn{2}{r}{Total}  & \multicolumn{1}{p{2cm}}{\hfill /35} \\
\end{tabular}


\vfill
AHT \today
\end{document}
